\documentclass[12pt]{article}

% ===== Shared settings =====
\usepackage[letterpaper,margin=0.7in]{geometry}
\renewcommand\normalsize{%
  \fontsize{13}{15}\selectfont
}
\usepackage{parskip,microtype,booktabs,enumitem,graphicx}
\usepackage[dvipsnames]{xcolor}
\usepackage{titlesec}
\usepackage{iftex}
\ifPDFTeX
  \usepackage[sfdefault]{carlito}
\else
  \usepackage{fontspec}
  \setmainfont{Georgia}
\fi
\usepackage{fancyhdr,lastpage}
\usepackage{hyperref}
\hypersetup{colorlinks=false,pdfborder={0 0 0}}

% ===== Colors =====
\definecolor{DeepBlue}{RGB}{0,40,90}
\definecolor{AccentGray}{gray}{0.25}

% ===== Headings =====
\titleformat{\section}{\Large\bfseries\color{DeepBlue}\raggedright}{}{0pt}{}
\titleformat{\subsection}{\normalsize\bfseries\color{DeepBlue}}{}{0pt}{}
\titlespacing*{\section}{0pt}{1.0em plus 0.2em minus 0.2em}{0.6em}
\titlespacing*{\subsection}{0pt}{0.8em plus 0.1em minus 0.1em}{0.4em}

% ===== Header/Footer and Metadata Macros =====
\newcommand{\CourseInfo}[2]{%
  \fancyhead[L]{\textbf{#1}}%
  \fancyhead[R]{\textbf{#2}}%
}

\newcommand{\StudyTitle}[1]{%
  {\LARGE\bfseries\color{DeepBlue} #1}\par\vspace{0.4em}\hrule\vspace{1em}
}

% ===== Footer =====
\pagestyle{fancy}
\fancyhf{}
\fancyfoot[C]{\textcolor{DeepBlue}{BIO 150 · Measurement, Scale, and Microscopy · Page \thepage\ of \pageref{LastPage}}}
\renewcommand{\headrulewidth}{0pt}
\renewcommand{\footrulewidth}{0.4pt}

% ===== Document Info =====
\CourseInfo{BIO 150 – Principles of Biology}{Lecture Recap Tues. Feb. 3, 2026}

\begin{document}

\StudyTitle{Lecture Recap — Measurement, Scale, and Microscopy}

This class period focused on building intuition for biological scale, practicing metric (SI) unit conversions, and introducing microscopy as a tool for visualizing cells. The emphasis was not speed, but confidence and accuracy when moving between units and understanding why scale matters in biology.


\section*{Metric System and SI Units}

Biology uses the International System of Units (SI), which is organized by powers of ten. Most conversions involve moving the decimal point in steps of three (factors of one thousand).

\begin{itemize}
  \item Large to small units: move the decimal to the right
  \item Small to large units: move the decimal to the left
  \item Common biology units include meters (m), millimeters (mm), micrometers ($\mu$m), and nanometers (nm)
\end{itemize}

\textbf{Centimeters and deciliters are common trouble spots} because they represent single decimal steps rather than thousand-fold changes. Most measurements we cover bypass these units entirely.

\section*{Thinking About Scale}

Students were asked to connect units to physical reference points:

\begin{itemize}
  \item A meter is approximately the length of a meter stick
  \item A kilometer is roughly comparable to a mile
  \item A millimeter is visible as a small mark on a ruler
  \item A micrometer and nanometer are far below unaided human vision
\end{itemize}

Developing intuition for scale is essential for understanding:

\begin{itemize}
  \item Cell size
  \item The limits of microscopy
  \item Diffusion
  \item Why cells cannot be arbitrarily large
\end{itemize}

\section*{Scientific Notation and Unit Accuracy}

Scientific notation is a compact way to represent very large or very small numbers and is commonly used in biology. Accuracy matters more than speed.

\textbf{Units must always be included with numerical answers.} A correct number without units is incomplete.

A real-world example discussed involved medical dosing: expressing quantities in appropriate units (such as milligrams rather than fractions of grams) reduces the risk of serious error.

\section*{Introduction to Microscopy}

A brief microscope demonstration introduced how microscopes allow visualization of structures too small to be seen with the naked eye.

Key concepts included:

\begin{itemize}
  \item Cells and tissues are three-dimensional, not flat
  \item Different structures come into focus at different depths (focal planes)
  \item Fine focus adjustments reveal details layer by layer
  \item Light intensity and condenser settings affect image clarity
\end{itemize}

Layered threads were used to demonstrate how focus changes with depth—an important concept when viewing real cells.

\section*{Historical Context: Discovery of Cells}

Early microscopy discoveries were reviewed:

\begin{itemize}
  \item Antonie van Leeuwenhoek observed living microorganisms
  \item Robert Hooke first described and named cells while examining cork
  \item Matthias Schleiden and Theodor Schwann concluded that plants and animals are composed of cells
\end{itemize}

These discoveries led to the development of cell theory.

\section*{Cell Theory and Limits on Cell Size}

Three core ideas of cell theory were emphasized:

\begin{enumerate}
  \item All living organisms are composed of one or more cells
  \item The cell is the basic unit of life
  \item All cells arise from pre-existing cells
\end{enumerate}

Cell size is limited by physical constraints:

\begin{itemize}
  \item Materials must enter and exit through the plasma membrane
  \item Diffusion depends on surface area, distance, and concentration gradients
  \item As cell size increases, volume increases faster than surface area
\end{itemize}

Some cells overcome these limits by being long and thin (for example, neurons), which reduces diffusion distance.
\newpage
\section*{Types of Microscopes}

Several microscopy approaches were briefly introduced:

\begin{itemize}
  \item Light microscopes use visible light and are limited by wavelength
  \item Electron microscopes offer much higher resolution but require non-living samples
  \item Specialized light techniques (phase contrast, fluorescence, confocal) improve contrast and resolution
\end{itemize}

Each method involves tradeoffs between resolution, sample preparation, and the ability to observe living cells.

\section*{Key Takeaways}

\begin{itemize}
  \item Biology depends on understanding scale and measurement
  \item Most metric conversions occur in factors of one thousand
  \item Accuracy and units matter more than speed
  \item Microscopes exist because cells are too small to see unaided
  \item Cell size is constrained by diffusion and surface-area-to-volume relationships
\end{itemize}

\end{document}
